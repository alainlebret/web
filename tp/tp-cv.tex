%%%%%%%%%%%%%%%%%%%%%%%%%%%%%%%%%%%%%%%%%%%%%%%%%%%%%%%%%%%%%%%%%%%%%%%%%%%%%%%
%                            CONCEPTION DE SITES WEB 
%  Cours proposé à l'ENSICAEN en spécialité informatique 1ère année (2014-2015)
%  puis au lycée Dorian de Paris en BTS SN-IR 1ère année (2015-2016)
%                                    2014-2015
%                                  Alain Lebret
% +++++++++++++++++++++++++++++++++++++++++++++++++++++++++++++++++++++++++++++
% Disponible sur GitHub : https://github.com/frosch74/web
% Mis à disposition selon les termes de la licence Creative Commons :
% CC BY-NC-SA 4.0 (http://creativecommons.org/licenses/by-nc-sa/4.0/
%%%%%%%%%%%%%%%%%%%%%%%%%%%%%%%%%%%%%%%%%%%%%%%%%%%%%%%%%%%%%%%%%%%%%%%%%%%%%%%
\documentclass[a4paper,12pt]{article}
\usepackage[frenchb]{babel}
\usepackage[T1]{fontenc}
%%%
\newif\ifTRUETYPE % Test pour l'utilisation des polices True Type comme Calibri utilisée dans la charte ENSICAEN
\TRUETYPEtrue
%\TRUETYPEfalse
\ifTRUETYPE
\else
  \usepackage[utf8]{inputenc}
\fi
%%%
\usepackage{graphicx}
\usepackage{colortbl}
\usepackage[table]{xcolor}
\usepackage[hmargin=3cm,bottom=5cm]{geometry}
\usepackage{fancyhdr}
\usepackage{amsmath,amsfonts,amssymb}
%%%%%%%%%%%%%%%%%%%%%%%%%%%%%%%%%%%%%%%%%%%%%%%%%%%%%%%%%%%%%%%%%%%%%%%%%%%%%%%
% Utilisation de polices True Type à la place des polices LaTeX traditionnelles
% Le package "inputenc" doit être enlevé et le fichier doit être compilé avec
% XeLaTeX
%%%%%%%%%%%%%%%%%%%%%%%%%%%%%%%%%%%%%%%%%%%%%%%%%%%%%%%%%%%%%%%%%%%%%%%%%%%%%%%
\ifTRUETYPE
\usepackage{fontspec}
\setmainfont{Calibri} 
\setmonofont[SmallCapsFont={Latin Modern Mono Caps}]{Latin Modern Mono Light}
\fi
%%%

\pagestyle{fancy}
\renewcommand{\headrule}{\relax}
\setlength{\headheight}{70pt}
\fancyhf{}
\lhead{}
\rhead{\includegraphics[height=\headheight]{logoEcole.jpg}}
\lfoot{\textit{A. Lebret}\\\includegraphics[height=0.52\headheight]{vague2013seule.jpg}
}
\rfoot{\thepage}

\definecolor{couleur_code}{RGB}{90,90,90} 
\newcommand\Code[1]{\textcolor{couleur_code}{\texttt{#1}}}
\newcounter{Question}
\setcounter{Question}{1}
\newcommand\Question{\vspace{5pt}\textbf{Question \arabic{Question}\addtocounter{Question}{1}.~~}}
\makeatletter
\newcommand{\Qlabel}[1]{\addtocounter{Question}{-1}\protected@write \@auxout {}{\string \newlabel {#1}{{\arabic{Question}}{}}}\addtocounter{Question}{1}}

\newcounter{Exercice}
\setcounter{Exercice}{1}
\newcommand\Exercice{\vspace{5pt}\textbf{Exercice \arabic{Exercice}\addtocounter{Exercice}{1}.~~}}
\makeatletter
\newcommand{\Qlabele}[1]{\addtocounter{Exercice}{-1}\protected@write \@auxout {}{\string \newlabel {#1}{{\arabic{Exercice}}{}}}\addtocounter{Exercice}{1}}
\makeatother

\newif\ifCORRECTION % Test pour l'affichage des corrections
%\CORRECTIONtrue
\CORRECTIONfalse

%%%%%%%%%%%%%%%%%%%%%%%%%%%%%%%%%%%%%%%%%%%%%%%%%%%%%%%%%%%%%%%%%%%%%%%%%%%%%%%
\begin{document}\parindent0pt\parskip\smallskipamount
%%%%%%%%%%%%%%%%%%%%%%%%%%%%%%%%%%%%%%%%%%%%%%%%%%%%%%%%%%%%%%%%%%%%%%%%%%%%%%%

\begin{center}
  \textbf{\large{Conception de sites Web - TP \no ~1}}\par\smallskip
  \textbf{Votre curriculum vit\ae~ en ligne avec HTML et CSS}
\end{center}\bigskip

\textbf{Objectif~:~} Dans ce TP, vous aurez à réalisez une page Web présentant
votre curriculum vit\ae ~et respectant les standards HTML5 et CSS3.

%%%%%%%%%%%%%%%%%%%%%%%%%%%%%%%%%%%%%%%%%%%%%%%%%%%%%%%%%%%%%%%%%%%%%%%%%%%%%%%
\section{Prise en main}
%%%%%%%%%%%%%%%%%%%%%%%%%%%%%%%%%%%%%%%%%%%%%%%%%%%%%%%%%%%%%%%%%%%%%%%%%%%%%%%
\`{A} la racine de votre répertoire utilisateur, vous disposez du 
sous-répertoire \Code{public\_html} dans lequel vous placerez les fichiers
de votre site Web personnel au fur et à mesure de son évolution. 
\Code{public\_html} correspondra donc au répertoire racine de votre site 
accessible depuis l'URL : \texttt{http://www.ecole.ensicaen.fr/\textasciitilde 
identifiant/}. 

Vérifiez tout d'abord que votre répertoire de travail \Code{public\_html}
est bien accessible à tout le monde en lecture, puis créez le sous-répertoire
\Code{styles} dans lequel vous placerez à l'avenir vos différentes feuilles
de styles.
  
\Exercice\\
Dans votre répertoire de travail, créez un fichier appelé \Code{page1.html},
par exemple avec le titre \og{}Une page minimaliste\fg{} (balise en paires 
\Code{<title>} dans l'en-tête) et contenant dans le corps le texte 
\og{}Bonjour à tous !\fg{} en titre de niveau 1 (balise en paires 
\Code{<h1>}). 
Vous pouvez également expérimenter d'autres balises que vous connaissez
(paragraphes, listes, etc.).

Validez votre page avec l'outil en ligne de l'organisme W3C que
l'on trouve à l'adresse \texttt{http://validator.w3.org}. 
La page sera considérée comme valide si l'outil affiche \og{}This document was
successfully checked as HTML5!\fg{}.
Visualisez alors votre page à l'aide d'un ou de plusieurs navigateurs (Firefox,
Chrome, Lynx, Internet Explorer, versions mobiles, etc.).

On rappelle qu'un document HTML5 minimal est constitué comme suit :
\begin{verbatim}
<!DOCTYPE html>
<html lang="fr">
   <head>
      <meta charset="utf-8" />
      <title>Un titre</title>
   </head>
   <body>
   </body>
</html>
\end{verbatim}

\Exercice\\
Réalisez une feuille de styles \Code{page1.css} que vous placerez dans le 
sous-répertoire \Code{styles} préalablement créé et que vous associerez à 
votre fichier \Code{page1.html}. 
Cette feuille de styles devra au minimum mettre en forme le titre de niveau
1 en modifiant la police de caractères, sa taille et sa couleur.
Votre feuille de style devra être validée sur l'outil en ligne 
\texttt{http://jigsaw.w3.org/css-validator/} puis testée en visualisant 
à nouveau \Code{page1.html} à l'aide d'un ou de plusieurs navigateurs.

%%%%%%%%%%%%%%%%%%%%%%%%%%%%%%%%%%%%%%%%%%%%%%%%%%%%%%%%%%%%%%%%%%%%%%%%%%%%%%%
\section{Création de votre CV en ligne}
%%%%%%%%%%%%%%%%%%%%%%%%%%%%%%%%%%%%%%%%%%%%%%%%%%%%%%%%%%%%%%%%%%%%%%%%%%%%%%%
La réalisation de votre curriculum vit\ae ~en ligne devra respecter les
standards HTML5 et CSS3.  

\subsection{Contenu du CV}

Les éléments que vous voudrez voir apparaître dans votre curriculum vit\ae 
~seront contenus dans le fichier \Code{cv\_fr.html} placé à la racine de votre
 site. Il devra être principalement structuré à l'aide des balises sémantiques
\Code{<header>}, \Code{<section>}, \Code{<article>} et \Code{<footer>} que 
nous avons rencontrées dans le cours. 
Les balises \texttt{<nav>} et \texttt{<aside>} seront quant à elles utilisées
lors les prochaines séances.

\Exercice\\
Proposez une structure HTML qui permettra de placer dans l'en-tête
 (\texttt{<header>}) vos prénom, nom, qualificatif vous décrivant
professionnellement, coordonnées, âge, voire une photographie. Le pied de page 
(\texttt{<footer>}) permettra d'indiquer des liens éventuels vers vos pages 
Facebook, Twitter, etc.. 
Le corps du curriculum vit\ae ~sera quant à lui constitué de six sections
(\Code{<section>}) correspondant à vos objectifs, votre formation, votre
expérience professionnelle, vos compétences (en informatique), les langues 
que vous parlez ainsi que vos intérêts et autres activités. Chaque section 
sera munie d'un titre de niveau 1 permettant de la décrire 
(\og{}prénom + nom\fg{}, \og{}objectifs\fg{}, \og{}formation\fg{}, etc.) et
éventuellement d'un titre de niveau 2, par exemple pour le qualificatif
professionnel ou bien les différents types d'autres activités (voir 
l'exemple de la figure ~\ref{fig:css1}).
Enfin, chaque formation ou expérience professionnelle sera décrite dans un 
article (\Code{<article>}) muni au moins d'un titre de niveau 1.
Validez votre document et visualisez le résultat.

Remarque 1 : Les dates et durées peuvent être judicieusement structurées à
l'aide de la balise \Code{<time>}.

Remarque 2 : La balise universelle \Code{<div>} ne devrait pas être employée
dans votre document. Un emploi minimal peut toutefois être accordé s'il est
justifié.

\subsection{Mise en forme du CV}
Il est temps de passer à la mise en forme de votre curriculum vit\ae ~en
utilisant les feuilles de styles CSS. 

\Exercice\\
Nous vous proposons de réaliser une feuille de styles \Code{cv1.css} que
vous placerez dans le sous-répertoire \Code{styles} préalablement créé et 
que vous associerez à votre fichier \Code{cv\-\_fr.html}. 
La mise en page attendue est du même type que celle qui vous est proposée 
sur la figure~\ref{fig:css1}. Les marges, espacements, couleurs, polices 
et mise en valeur du texte sont de votre ressort en accord avec vos goûts.

Validez votre feuille de styles et visualisez votre curriculum vit\ae 
~depuis différents navigateurs.

\begin{figure}[h]
  \begin{center}
    \includegraphics[width=0.85\textwidth]{pictures/image-cv1-css.png} 
    \caption{Exemple de mise en forme avec \Code{cv1.css}}
    \label{fig:css1} 
  \end{center}
\end{figure}

\subsection{Changer la mise en forme}
L'intérêt de dissocier le fond de la forme, est que vous devriez pouvoir
assez facilement changer la mise en page de votre curriculum vit\ae ~en 
remplaçant la feuille de styles précédente.

\Exercice\\
Nous vous proposons de réaliser la feuille de styles \Code{cv2.css} que 
vous associerez à votre fichier \Code{cv\_fr.html}. 
La mise en page attendue sera du même type que celle qui vous est proposée
sur la figure~\ref{fig:css2}.

Validez votre feuille de styles et visualisez votre curriculum vit\ae 
~depuis différents navigateurs.


\subsection{Pourquoi pas une version anglaise ?}

\Exercice\\ 
Traduisez votre curriculum vit\ae ~en anglais et nommez votre fichier 
\Code{cv\_en.html}. Modifiez le pied de page de votre structure HTML de
manière à insérer les images réduites de drapeaux français et anglais
auxquels vous associerez à un lien vers les pages correspondantes. 

\begin{figure}[h]
  \begin{center}
    \includegraphics[width=1.0\textwidth]{pictures/image-cv2-css.png} 
    \caption{Exemple de mise en forme avec \Code{cv2.css}}
    \label{fig:css2} 
  \end{center}
\end{figure}

\end{document}
